% Preamble
\documentclass[12pt]{simple_doc}

% Packages
\usepackage{amsmath}
\usepackage{amssymb}
\usepackage{simple_note}
\usepackage{enumitem}
\usepackage{tkz-euclide}

\hypersetup{
    colorlinks=true,
    linkcolor=cyan,
    filecolor=blue,
    urlcolor=magenta,
}

% Document
\begin{document}
    \exerheader{Statistics}{PDF of RV Summation}{\today}{ctbBlueDark}

    \begin{cbstripe}{Problem}{ctbGreenDark}{ctbGreenLight}
        \textit{Given two random variables $X$ and $Y$ over uniform distribution $U(0, 1)$. Find
        the PDF of $Z = X + Y$.}
    \end{cbstripe}

    Since $X, Y \sim U(0, 1)$, the PDFs of $X$ and $Y$ are $f_X(x) = 1$ for $0 < x < 1$ and
    $f_Y(y) = 1$ for $0 < y < 1$. (Properties of uniform distribution is listed in
    \href{https://en.wikipedia.org/wiki/Continuous_uniform_distribution}{wikipedia})

    Then
    \begin{equation} \label{eq:1}
		\begin{aligned}
		f_Z(z) &= P(Z = z) = P(X + Y = z) = \sum_x P(X = x, Y = z - x) \\
		   &= \int_{-\infty}^{\infty} f_X(x)f_Y(z-x)dx\\
		\end{aligned}
	\end{equation}
    Since $f_X$ and $f_Y$ have nonzero values only in $(0, 1)$,
    \begin{equation}
        0 < x < 1, \ and \ 0 < z - x < 1 \ or \ z - 1 < x < z
    \end{equation}
    To combine these 2 conditions to find common ranges for x with nonzero $f_X$ and $f_Y$, there
    are 2 cases:
    \begin{enumerate}
        \item When $0 < z < 1$, we need to choose 0 over $z-1$ on the left side and $z$ over 1 on the
              right side. So
              \begin{equation}
                  f_Z(z) = \int_{0}^{z} f_X(x)f_Y(z-x)dx\\
                         = \int_{0}^{z} 1 dx = z
              \end{equation}
        \item When $1 < z < 2$, we need to choose $z-1$ over 0 on the left side and 1 over $z$ on the
              right side. So
              \begin{equation}
                  f_Z(z) = \int_{z-1}^{1} f_X(x)f_Y(z-x)dx\\
                         = \int_{z-1}^{1} 1 dx = 2 - z
              \end{equation}
    \end{enumerate}

    Therefore,
    \begin{equation}
        f_Z(z) =
            \begin{cases}
              z, & \text{if}\ 0 < z \leqslant 1\\
              2 - z, & \text{if}\ 1 < z < 2\\
              0, & \text{otherwise}
            \end{cases}
    \end{equation}

    The graph is 2 line segments

    \begin{center}
    \begin{tikzpicture}[x=3cm,y=3cm,scale=1]
       \tkzInit[xmax=2.1,ymax=1.1,xmin=0,ymin=0]
       \tkzAxeXY
       \draw[thick] (0, 0) -- (1, 1);
       \draw[thick] (1, 1) -- (2, 0);
    \end{tikzpicture}
    \end{center}


    This \href{http://www.statisticalengineering.com/sums_of_random_variables.htm}{statisticalengineering.com post}
    and \href{https://math.stackexchange.com/questions/357672/density-of-sum-of-two-independent-uniform-random-variables-on-0-1}{stackoverflow discussion}
    discusses the same problem. For sum of $n$ variables, see
    \href{https://en.wikipedia.org/wiki/Irwin%E2%80%93Hall_distribution}{ Irwin-Hall Distribution}.

    This \href{https://towardsdatascience.com/sum-of-exponential-random-variables-b023b61f0c0f}{post}
    discusses the exponential distribution case. If $X, Y \sim Exp(\lambda)$, the Equation (\ref{eq:1})
    becomes

    \begin{equation} \label{eq:2}
		\begin{aligned}
		f_Z(z) &= \int_{-\infty}^{\infty} f_X(x)f_Y(z-x)dx\\
               &= \int_{0}^{z} \lambda e^{-\lambda x} \lambda e^{-\lambda (z - x)} dx\\
               &= \lambda^2 \int_{0}^{z} e^{-\lambda z} dx\\
               &= \lambda^2 z e^{-\lambda z}
		\end{aligned}
	\end{equation}

    where $x$ and $z-x$ have to be positive, which means $0 < x < z$.

    More sum over other distributions are discussed
    \href{http://www.dim.uchile.cl/~mkiwi/ma34a/libro/ch7.pdf}{here}.

\end{document}
